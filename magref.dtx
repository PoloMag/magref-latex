% \iffalse meta-comment
%
% Copyright (C) 2014 by Fábio Fortkamp <fabio@fabiofortkamp.com>
%
% This work may be distributed and/or modified under the
% conditions of the LaTeX Project Public License, either version 1.3
% of this license or (at your option) any later version.
% The latest version of this license is in
%   http://www.latex-project.org/lppl.txt
% and version 1.3 or later is part of all distributions of LaTeX
% version 2005/12/01 or later.
%
% This work has the LPPL maintenance status `maintained'.
%
% The Current Maintainer of this work is Fábio Fortkamp.
%
% This work consists of the files magref.dtx and magref.ins
% and the derived file magref.sty and magref.pdf.
%
% \fi
%
% \iffalse
%<*driver>
\ProvidesFile{magref.dtx}
%</driver>
%<package>\NeedsTeXFormat{LaTeX2e}
%<package>\ProvidesPackage{magref}[2016/06/29 v0.3.1]
%<package>
%<package>\RequirePackage{engsymbols}
%<package>\RequirePackage[version=4]{mhchem}
%<package>\RequirePackage[per-mode=symbol]{siunitx}
%<package>\RequirePackage{ifthen}
%<package>
%<*driver>
\documentclass{ltxdoc}
\usepackage{magref}[2016/05/27]
\usepackage[utf8]{inputenc}
\usepackage[T1]{fontenc}
\bibliographystyle{plain}
\EnableCrossrefs         
\CodelineIndex
\RecordChanges
\begin{document}
  \DocInput{magref.dtx}
  \PrintChanges
  \PrintIndex
\end{document}
%</driver>
% \fi
%
% \CheckSum{0}
%
% \CharacterTable
%  {Upper-case    \A\B\C\D\E\F\G\H\I\J\K\L\M\N\O\P\Q\R\S\T\U\V\W\X\Y\Z
%   Lower-case    \a\b\c\d\e\f\g\h\i\j\k\l\m\n\o\p\q\r\s\t\u\v\w\x\y\z
%   Digits        \0\1\2\3\4\5\6\7\8\9
%   Exclamation   \!     Double quote  \"     Hash (number) \#
%   Dollar        \$     Percent       \%     Ampersand     \&
%   Acute accent  \'     Left paren    \(     Right paren   \)
%   Asterisk      \*     Plus          \+     Comma         \,
%   Minus         \-     Point         \.     Solidus       \/
%   Colon         \:     Semicolon     \;     Less than     \<
%   Equals        \=     Greater than  \>     Question mark \?
%   Commercial at \@     Left bracket  \[     Backslash     \\
%   Right bracket \]     Circumflex    \^     Underscore    \_
%   Grave accent  \`     Left brace    \{     Vertical bar  \|
%   Right brace   \}     Tilde         \~}
%
%
% \changes{v0.1}{2016/03/08}{Initial version}
% \changes{v0.2}{2016/03/21}{Included magnetic dipole-related vector fields}
% \changes{v0.3}{2016/05/27}{Included more commands based on frequency of use in papers}
% \changes{v0.3.1}{2016/06/29}{Capitalized subscripts of power}
%
% \GetFileInfo{magref.sty}
%
% \DoNotIndex{\newcommand,\newenvironment}
% 
%
% \title{The \textsf{magref} package\thanks{This document
%   corresponds to \textsf{magref}~\fileversion, dated \filedate.}}
% \author{Fábio Fortkamp \\ \texttt{fabio@fabiofortkamp.com}}
%
% \maketitle
%
% \section{Introduction}
% \label{sec:introduction}
%
% This document describes the \textsf{magref} package, a collection of macros to facilitate the writing of papers and reports on magnetic refrigeration.
% It just defines common macros and loads packages that are normally used:
%
% \begin{itemize}
% \item \textsf{engsymbols}
% \item \textsf{sinuitx}
% \item \textsf{mhchem}
% \item \textsf{ifthen}
% \end{itemize}
%
% 
% Please notice that the user should refer to other references such as papers and textbooks to get the meaning of the symbols I describe here.
%
% \section{Implementation}
%
% The use of the conditional commands to define these custom macro is because some packages and classes that I use in conjuntion with \textsf{magref} provide some obscure commands that clash with them. A normal user should not have any problems with that.
%
% \subsection{Specfic heats}
% \label{sec:basic-operations}
%
%    \begin{macrocode}
\newcommand{\cpo}{c_{p,0}}
\newcommand{\cvo}{c_{v,0}}
%    \end{macrocode}
%
%\subsection{Common refrigeration and thermodynamic parameters}
%\label{sec:ref}
% 
%    \begin{macrocode}
\newcommand{\qe}{\rate{Q}\ped{e}}
\newcommand{\dtspan}{\Delta{}T\ped{span}}
\newcommand{\w}{\rate{W}}
\newcommand{\wpump}{\w\ped{P}}
\newcommand{\wmag}{\w\ped{M}}
\newcommand{\wvisc}{\w\ped{visc}}
\newcommand{\cop}{\mathrm{COP}}
\newcommand{\dtad}{\Delta{}T\ped{ad}}
\newcommand{\dsm}{\Delta{}s\ped{M}}
\newcommand{\tc}{T\ped{C}}
\ifthenelse{\isundefined{\th}}{\newcommand{\th}{T\ped{H}}}{\renewcommand{\th}{T\ped{H}}}
\newcommand{\qc}{\rate{Q}\ped{C}}
\newcommand{\qh}{\rate{Q}\ped{H}}
\newcommand{\ths}{T\ped{H,s}}
\newcommand{\tcs}{T\ped{C,s}}
\newcommand{\lcool}{\Lambda\ped{cool}}
\newcommand{\tcurie}{T\ped{Curie}}
\newcommand{\ntu}{\mathrm{NTU}}
\newcommand{\mf}{\rate{m}\ped{f}}
\newcommand{\vvf}{\rate{\vv}\ped{f}}
%    \end{macrocode}
%
%\subsection{Common vector fields}
%\label{sec:vectors}
%
%    \begin{macrocode}
\newcommand{\rvec}{\nvector{r}}
\newcommand{\nvh}{\nvector{H}}
\newcommand{\nvb}{\nvector{B}}
\newcommand{\nva}{\nvector{A}}
\newcommand{\nvrem}{\nvector{B}\ped{rem}}
\newcommand{\nvbrem}{\nvrem}
\newcommand{\nvm}{\nvector{M}}
\newcommand{\nvdip}{\nvector{m}}
\newcommand{\nvnetdip}{\nvector{\mathcal{M}}}
\newcommand{\nvbremhat}{\hat{\nvector{B}}\ped{rem}}
\newcommand{\nvbi}{\nvector{B}_k}
\newcommand{\nvhi}{\nvector{H}_k}
\newcommand{\nvremi}{\nvector{B}_{\mathrm{rem},k}}
\newcommand{\nvbremi}{\nvremi}
\newcommand{\nvai}{\nvector{A}_k}
\newcommand{\nvha}{\nvh\ped{a}}
\newcommand{\nvhd}{\nvh\ped{d}}
%    \end{macrocode}
%
%\subsection{Scalar fields defined from vector fields}
%\label{sec:scalar-from-vector}
%
%    \begin{macrocode}
\newcommand{\nha}{H\ped{a}}
%    \end{macrocode}
%
%\subsection{Common other scalar parameters}
%\label{sec:scalar}
%
%    \begin{macrocode}
\newcommand{\mur}{\mu\ped{r}}
\newcommand{\bl}{B\ped{l}}
\newcommand{\bh}{B\ped{h}}
\newcommand{\hal}{H\ped{a,l}}
\newcommand{\hah}{H\ped{a,h}}
\newcommand{\brem}{B\ped{rem}}
\newcommand{\muri}{\mu_{\mathrm{r},k}}
\newcommand{\bremi}{B_{\mathrm{rem},k}}
\newcommand{\avgb}[1]{\left\langle B^{2/3} \right\rangle\ped{#1}}
\newcommand{\qmce}{q'\ped{MCE}}
%    \end{macrocode}
%
%\subsection{Aliases from common terms}
%\label{sec:term}
%
%    \begin{macrocode}
\newcommand{\ndfeb}{\ce{Nd{-}Fe{-}B} }
%    \end{macrocode}
%
%\subsection{Thermodynamic potentials}
%\label{sec:potential}
%
%    \begin{macrocode}
\newcommand{\sigmaxdh}{\sigma \diffd{H\ped{a}}}
\newcommand{\hxdsigma}{H\ped{a}\diffd{\sigma}}
%    \end{macrocode}
%
%\subsection{Geometric parameters}
%\label{sec:geom}
%
%    \begin{macrocode}
\newcommand{\ri}{R\ped{i}}
\ifthenelse{\isundefined{\ro}}{\newcommand{\ro}{R\ped{o}}}{\renewcommand{\ro}{R\ped{o}}}
\newcommand{\rg}{R\ped{g}}
\newcommand{\rs}{R\ped{s}}
\newcommand{\rc}{R\ped{c}}
\newcommand{\re}{R\ped{e}}
\newcommand{\hgap}{h\ped{gap}}
%    \end{macrocode}
%
%\subsection{Coefficients for the analytical solution}
%\label{sec:coef}
%
%    \begin{macrocode}
\newcommand{\acoef}[2]{a_{{#1},\mathrm{#2}}}
\newcommand{\bcoef}[2]{b_{{#1},\mathrm{#2}}}
\newcommand{\Azexpr}[1]{A_{\mathrm{#1},z}}
\newcommand{\bremii}{B_{\mathrm{rem,II}}}
\newcommand{\bremiv}{B_{\mathrm{rem,IV}}}
\newcommand{\murn}[1]{\mu\ped{r,#1}}
\newcommand{\aIII}{\acoef{1}{III}}
\newcommand{\bIII}{\bcoef{1}{III}}
\newcommand{\nvbIII}{\nvector{B}\ped{III}}
\newcommand{\BIII}{B\ped{III}}
%    \end{macrocode}
%
% \Finale
\endinput
